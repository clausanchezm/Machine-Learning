\documentclass[11pt]{article}

    \usepackage[breakable]{tcolorbox}
    \usepackage{parskip} % Stop auto-indenting (to mimic markdown behaviour)
    
    \usepackage{iftex}
    \ifPDFTeX
    	\usepackage[T1]{fontenc}
    	\usepackage{mathpazo}
    \else
    	\usepackage{fontspec}
    \fi

    % Basic figure setup, for now with no caption control since it's done
    % automatically by Pandoc (which extracts ![](path) syntax from Markdown).
    \usepackage{graphicx}
    % Maintain compatibility with old templates. Remove in nbconvert 6.0
    \let\Oldincludegraphics\includegraphics
    % Ensure that by default, figures have no caption (until we provide a
    % proper Figure object with a Caption API and a way to capture that
    % in the conversion process - todo).
    \usepackage{caption}
    \DeclareCaptionFormat{nocaption}{}
    \captionsetup{format=nocaption,aboveskip=0pt,belowskip=0pt}

    \usepackage{float}
    \floatplacement{figure}{H} % forces figures to be placed at the correct location
    \usepackage{xcolor} % Allow colors to be defined
    \usepackage{enumerate} % Needed for markdown enumerations to work
    \usepackage{geometry} % Used to adjust the document margins
    \usepackage{amsmath} % Equations
    \usepackage{amssymb} % Equations
    \usepackage{textcomp} % defines textquotesingle
    % Hack from http://tex.stackexchange.com/a/47451/13684:
    \AtBeginDocument{%
        \def\PYZsq{\textquotesingle}% Upright quotes in Pygmentized code
    }
    \usepackage{upquote} % Upright quotes for verbatim code
    \usepackage{eurosym} % defines \euro
    \usepackage[mathletters]{ucs} % Extended unicode (utf-8) support
    \usepackage{fancyvrb} % verbatim replacement that allows latex
    \usepackage{grffile} % extends the file name processing of package graphics 
                         % to support a larger range
    \makeatletter % fix for old versions of grffile with XeLaTeX
    \@ifpackagelater{grffile}{2019/11/01}
    {
      % Do nothing on new versions
    }
    {
      \def\Gread@@xetex#1{%
        \IfFileExists{"\Gin@base".bb}%
        {\Gread@eps{\Gin@base.bb}}%
        {\Gread@@xetex@aux#1}%
      }
    }
    \makeatother
    \usepackage[Export]{adjustbox} % Used to constrain images to a maximum size
    \adjustboxset{max size={0.9\linewidth}{0.9\paperheight}}

    % The hyperref package gives us a pdf with properly built
    % internal navigation ('pdf bookmarks' for the table of contents,
    % internal cross-reference links, web links for URLs, etc.)
    \usepackage{hyperref}
    % The default LaTeX title has an obnoxious amount of whitespace. By default,
    % titling removes some of it. It also provides customization options.
    \usepackage{titling}
    \usepackage{longtable} % longtable support required by pandoc >1.10
    \usepackage{booktabs}  % table support for pandoc > 1.12.2
    \usepackage[inline]{enumitem} % IRkernel/repr support (it uses the enumerate* environment)
    \usepackage[normalem]{ulem} % ulem is needed to support strikethroughs (\sout)
                                % normalem makes italics be italics, not underlines
    \usepackage{mathrsfs}
    

    
    % Colors for the hyperref package
    \definecolor{urlcolor}{rgb}{0,.145,.698}
    \definecolor{linkcolor}{rgb}{.71,0.21,0.01}
    \definecolor{citecolor}{rgb}{.12,.54,.11}

    % ANSI colors
    \definecolor{ansi-black}{HTML}{3E424D}
    \definecolor{ansi-black-intense}{HTML}{282C36}
    \definecolor{ansi-red}{HTML}{E75C58}
    \definecolor{ansi-red-intense}{HTML}{B22B31}
    \definecolor{ansi-green}{HTML}{00A250}
    \definecolor{ansi-green-intense}{HTML}{007427}
    \definecolor{ansi-yellow}{HTML}{DDB62B}
    \definecolor{ansi-yellow-intense}{HTML}{B27D12}
    \definecolor{ansi-blue}{HTML}{208FFB}
    \definecolor{ansi-blue-intense}{HTML}{0065CA}
    \definecolor{ansi-magenta}{HTML}{D160C4}
    \definecolor{ansi-magenta-intense}{HTML}{A03196}
    \definecolor{ansi-cyan}{HTML}{60C6C8}
    \definecolor{ansi-cyan-intense}{HTML}{258F8F}
    \definecolor{ansi-white}{HTML}{C5C1B4}
    \definecolor{ansi-white-intense}{HTML}{A1A6B2}
    \definecolor{ansi-default-inverse-fg}{HTML}{FFFFFF}
    \definecolor{ansi-default-inverse-bg}{HTML}{000000}

    % common color for the border for error outputs.
    \definecolor{outerrorbackground}{HTML}{FFDFDF}

    % commands and environments needed by pandoc snippets
    % extracted from the output of `pandoc -s`
    \providecommand{\tightlist}{%
      \setlength{\itemsep}{0pt}\setlength{\parskip}{0pt}}
    \DefineVerbatimEnvironment{Highlighting}{Verbatim}{commandchars=\\\{\}}
    % Add ',fontsize=\small' for more characters per line
    \newenvironment{Shaded}{}{}
    \newcommand{\KeywordTok}[1]{\textcolor[rgb]{0.00,0.44,0.13}{\textbf{{#1}}}}
    \newcommand{\DataTypeTok}[1]{\textcolor[rgb]{0.56,0.13,0.00}{{#1}}}
    \newcommand{\DecValTok}[1]{\textcolor[rgb]{0.25,0.63,0.44}{{#1}}}
    \newcommand{\BaseNTok}[1]{\textcolor[rgb]{0.25,0.63,0.44}{{#1}}}
    \newcommand{\FloatTok}[1]{\textcolor[rgb]{0.25,0.63,0.44}{{#1}}}
    \newcommand{\CharTok}[1]{\textcolor[rgb]{0.25,0.44,0.63}{{#1}}}
    \newcommand{\StringTok}[1]{\textcolor[rgb]{0.25,0.44,0.63}{{#1}}}
    \newcommand{\CommentTok}[1]{\textcolor[rgb]{0.38,0.63,0.69}{\textit{{#1}}}}
    \newcommand{\OtherTok}[1]{\textcolor[rgb]{0.00,0.44,0.13}{{#1}}}
    \newcommand{\AlertTok}[1]{\textcolor[rgb]{1.00,0.00,0.00}{\textbf{{#1}}}}
    \newcommand{\FunctionTok}[1]{\textcolor[rgb]{0.02,0.16,0.49}{{#1}}}
    \newcommand{\RegionMarkerTok}[1]{{#1}}
    \newcommand{\ErrorTok}[1]{\textcolor[rgb]{1.00,0.00,0.00}{\textbf{{#1}}}}
    \newcommand{\NormalTok}[1]{{#1}}
    
    % Additional commands for more recent versions of Pandoc
    \newcommand{\ConstantTok}[1]{\textcolor[rgb]{0.53,0.00,0.00}{{#1}}}
    \newcommand{\SpecialCharTok}[1]{\textcolor[rgb]{0.25,0.44,0.63}{{#1}}}
    \newcommand{\VerbatimStringTok}[1]{\textcolor[rgb]{0.25,0.44,0.63}{{#1}}}
    \newcommand{\SpecialStringTok}[1]{\textcolor[rgb]{0.73,0.40,0.53}{{#1}}}
    \newcommand{\ImportTok}[1]{{#1}}
    \newcommand{\DocumentationTok}[1]{\textcolor[rgb]{0.73,0.13,0.13}{\textit{{#1}}}}
    \newcommand{\AnnotationTok}[1]{\textcolor[rgb]{0.38,0.63,0.69}{\textbf{\textit{{#1}}}}}
    \newcommand{\CommentVarTok}[1]{\textcolor[rgb]{0.38,0.63,0.69}{\textbf{\textit{{#1}}}}}
    \newcommand{\VariableTok}[1]{\textcolor[rgb]{0.10,0.09,0.49}{{#1}}}
    \newcommand{\ControlFlowTok}[1]{\textcolor[rgb]{0.00,0.44,0.13}{\textbf{{#1}}}}
    \newcommand{\OperatorTok}[1]{\textcolor[rgb]{0.40,0.40,0.40}{{#1}}}
    \newcommand{\BuiltInTok}[1]{{#1}}
    \newcommand{\ExtensionTok}[1]{{#1}}
    \newcommand{\PreprocessorTok}[1]{\textcolor[rgb]{0.74,0.48,0.00}{{#1}}}
    \newcommand{\AttributeTok}[1]{\textcolor[rgb]{0.49,0.56,0.16}{{#1}}}
    \newcommand{\InformationTok}[1]{\textcolor[rgb]{0.38,0.63,0.69}{\textbf{\textit{{#1}}}}}
    \newcommand{\WarningTok}[1]{\textcolor[rgb]{0.38,0.63,0.69}{\textbf{\textit{{#1}}}}}
    
    
    % Define a nice break command that doesn't care if a line doesn't already
    % exist.
    \def\br{\hspace*{\fill} \\* }
    % Math Jax compatibility definitions
    \def\gt{>}
    \def\lt{<}
    \let\Oldtex\TeX
    \let\Oldlatex\LaTeX
    \renewcommand{\TeX}{\textrm{\Oldtex}}
    \renewcommand{\LaTeX}{\textrm{\Oldlatex}}
    % Document parameters
    % Document title
    \title{1ML}
    
    
    
    
    
% Pygments definitions
\makeatletter
\def\PY@reset{\let\PY@it=\relax \let\PY@bf=\relax%
    \let\PY@ul=\relax \let\PY@tc=\relax%
    \let\PY@bc=\relax \let\PY@ff=\relax}
\def\PY@tok#1{\csname PY@tok@#1\endcsname}
\def\PY@toks#1+{\ifx\relax#1\empty\else%
    \PY@tok{#1}\expandafter\PY@toks\fi}
\def\PY@do#1{\PY@bc{\PY@tc{\PY@ul{%
    \PY@it{\PY@bf{\PY@ff{#1}}}}}}}
\def\PY#1#2{\PY@reset\PY@toks#1+\relax+\PY@do{#2}}

\@namedef{PY@tok@w}{\def\PY@tc##1{\textcolor[rgb]{0.73,0.73,0.73}{##1}}}
\@namedef{PY@tok@c}{\let\PY@it=\textit\def\PY@tc##1{\textcolor[rgb]{0.25,0.50,0.50}{##1}}}
\@namedef{PY@tok@cp}{\def\PY@tc##1{\textcolor[rgb]{0.74,0.48,0.00}{##1}}}
\@namedef{PY@tok@k}{\let\PY@bf=\textbf\def\PY@tc##1{\textcolor[rgb]{0.00,0.50,0.00}{##1}}}
\@namedef{PY@tok@kp}{\def\PY@tc##1{\textcolor[rgb]{0.00,0.50,0.00}{##1}}}
\@namedef{PY@tok@kt}{\def\PY@tc##1{\textcolor[rgb]{0.69,0.00,0.25}{##1}}}
\@namedef{PY@tok@o}{\def\PY@tc##1{\textcolor[rgb]{0.40,0.40,0.40}{##1}}}
\@namedef{PY@tok@ow}{\let\PY@bf=\textbf\def\PY@tc##1{\textcolor[rgb]{0.67,0.13,1.00}{##1}}}
\@namedef{PY@tok@nb}{\def\PY@tc##1{\textcolor[rgb]{0.00,0.50,0.00}{##1}}}
\@namedef{PY@tok@nf}{\def\PY@tc##1{\textcolor[rgb]{0.00,0.00,1.00}{##1}}}
\@namedef{PY@tok@nc}{\let\PY@bf=\textbf\def\PY@tc##1{\textcolor[rgb]{0.00,0.00,1.00}{##1}}}
\@namedef{PY@tok@nn}{\let\PY@bf=\textbf\def\PY@tc##1{\textcolor[rgb]{0.00,0.00,1.00}{##1}}}
\@namedef{PY@tok@ne}{\let\PY@bf=\textbf\def\PY@tc##1{\textcolor[rgb]{0.82,0.25,0.23}{##1}}}
\@namedef{PY@tok@nv}{\def\PY@tc##1{\textcolor[rgb]{0.10,0.09,0.49}{##1}}}
\@namedef{PY@tok@no}{\def\PY@tc##1{\textcolor[rgb]{0.53,0.00,0.00}{##1}}}
\@namedef{PY@tok@nl}{\def\PY@tc##1{\textcolor[rgb]{0.63,0.63,0.00}{##1}}}
\@namedef{PY@tok@ni}{\let\PY@bf=\textbf\def\PY@tc##1{\textcolor[rgb]{0.60,0.60,0.60}{##1}}}
\@namedef{PY@tok@na}{\def\PY@tc##1{\textcolor[rgb]{0.49,0.56,0.16}{##1}}}
\@namedef{PY@tok@nt}{\let\PY@bf=\textbf\def\PY@tc##1{\textcolor[rgb]{0.00,0.50,0.00}{##1}}}
\@namedef{PY@tok@nd}{\def\PY@tc##1{\textcolor[rgb]{0.67,0.13,1.00}{##1}}}
\@namedef{PY@tok@s}{\def\PY@tc##1{\textcolor[rgb]{0.73,0.13,0.13}{##1}}}
\@namedef{PY@tok@sd}{\let\PY@it=\textit\def\PY@tc##1{\textcolor[rgb]{0.73,0.13,0.13}{##1}}}
\@namedef{PY@tok@si}{\let\PY@bf=\textbf\def\PY@tc##1{\textcolor[rgb]{0.73,0.40,0.53}{##1}}}
\@namedef{PY@tok@se}{\let\PY@bf=\textbf\def\PY@tc##1{\textcolor[rgb]{0.73,0.40,0.13}{##1}}}
\@namedef{PY@tok@sr}{\def\PY@tc##1{\textcolor[rgb]{0.73,0.40,0.53}{##1}}}
\@namedef{PY@tok@ss}{\def\PY@tc##1{\textcolor[rgb]{0.10,0.09,0.49}{##1}}}
\@namedef{PY@tok@sx}{\def\PY@tc##1{\textcolor[rgb]{0.00,0.50,0.00}{##1}}}
\@namedef{PY@tok@m}{\def\PY@tc##1{\textcolor[rgb]{0.40,0.40,0.40}{##1}}}
\@namedef{PY@tok@gh}{\let\PY@bf=\textbf\def\PY@tc##1{\textcolor[rgb]{0.00,0.00,0.50}{##1}}}
\@namedef{PY@tok@gu}{\let\PY@bf=\textbf\def\PY@tc##1{\textcolor[rgb]{0.50,0.00,0.50}{##1}}}
\@namedef{PY@tok@gd}{\def\PY@tc##1{\textcolor[rgb]{0.63,0.00,0.00}{##1}}}
\@namedef{PY@tok@gi}{\def\PY@tc##1{\textcolor[rgb]{0.00,0.63,0.00}{##1}}}
\@namedef{PY@tok@gr}{\def\PY@tc##1{\textcolor[rgb]{1.00,0.00,0.00}{##1}}}
\@namedef{PY@tok@ge}{\let\PY@it=\textit}
\@namedef{PY@tok@gs}{\let\PY@bf=\textbf}
\@namedef{PY@tok@gp}{\let\PY@bf=\textbf\def\PY@tc##1{\textcolor[rgb]{0.00,0.00,0.50}{##1}}}
\@namedef{PY@tok@go}{\def\PY@tc##1{\textcolor[rgb]{0.53,0.53,0.53}{##1}}}
\@namedef{PY@tok@gt}{\def\PY@tc##1{\textcolor[rgb]{0.00,0.27,0.87}{##1}}}
\@namedef{PY@tok@err}{\def\PY@bc##1{{\setlength{\fboxsep}{\string -\fboxrule}\fcolorbox[rgb]{1.00,0.00,0.00}{1,1,1}{\strut ##1}}}}
\@namedef{PY@tok@kc}{\let\PY@bf=\textbf\def\PY@tc##1{\textcolor[rgb]{0.00,0.50,0.00}{##1}}}
\@namedef{PY@tok@kd}{\let\PY@bf=\textbf\def\PY@tc##1{\textcolor[rgb]{0.00,0.50,0.00}{##1}}}
\@namedef{PY@tok@kn}{\let\PY@bf=\textbf\def\PY@tc##1{\textcolor[rgb]{0.00,0.50,0.00}{##1}}}
\@namedef{PY@tok@kr}{\let\PY@bf=\textbf\def\PY@tc##1{\textcolor[rgb]{0.00,0.50,0.00}{##1}}}
\@namedef{PY@tok@bp}{\def\PY@tc##1{\textcolor[rgb]{0.00,0.50,0.00}{##1}}}
\@namedef{PY@tok@fm}{\def\PY@tc##1{\textcolor[rgb]{0.00,0.00,1.00}{##1}}}
\@namedef{PY@tok@vc}{\def\PY@tc##1{\textcolor[rgb]{0.10,0.09,0.49}{##1}}}
\@namedef{PY@tok@vg}{\def\PY@tc##1{\textcolor[rgb]{0.10,0.09,0.49}{##1}}}
\@namedef{PY@tok@vi}{\def\PY@tc##1{\textcolor[rgb]{0.10,0.09,0.49}{##1}}}
\@namedef{PY@tok@vm}{\def\PY@tc##1{\textcolor[rgb]{0.10,0.09,0.49}{##1}}}
\@namedef{PY@tok@sa}{\def\PY@tc##1{\textcolor[rgb]{0.73,0.13,0.13}{##1}}}
\@namedef{PY@tok@sb}{\def\PY@tc##1{\textcolor[rgb]{0.73,0.13,0.13}{##1}}}
\@namedef{PY@tok@sc}{\def\PY@tc##1{\textcolor[rgb]{0.73,0.13,0.13}{##1}}}
\@namedef{PY@tok@dl}{\def\PY@tc##1{\textcolor[rgb]{0.73,0.13,0.13}{##1}}}
\@namedef{PY@tok@s2}{\def\PY@tc##1{\textcolor[rgb]{0.73,0.13,0.13}{##1}}}
\@namedef{PY@tok@sh}{\def\PY@tc##1{\textcolor[rgb]{0.73,0.13,0.13}{##1}}}
\@namedef{PY@tok@s1}{\def\PY@tc##1{\textcolor[rgb]{0.73,0.13,0.13}{##1}}}
\@namedef{PY@tok@mb}{\def\PY@tc##1{\textcolor[rgb]{0.40,0.40,0.40}{##1}}}
\@namedef{PY@tok@mf}{\def\PY@tc##1{\textcolor[rgb]{0.40,0.40,0.40}{##1}}}
\@namedef{PY@tok@mh}{\def\PY@tc##1{\textcolor[rgb]{0.40,0.40,0.40}{##1}}}
\@namedef{PY@tok@mi}{\def\PY@tc##1{\textcolor[rgb]{0.40,0.40,0.40}{##1}}}
\@namedef{PY@tok@il}{\def\PY@tc##1{\textcolor[rgb]{0.40,0.40,0.40}{##1}}}
\@namedef{PY@tok@mo}{\def\PY@tc##1{\textcolor[rgb]{0.40,0.40,0.40}{##1}}}
\@namedef{PY@tok@ch}{\let\PY@it=\textit\def\PY@tc##1{\textcolor[rgb]{0.25,0.50,0.50}{##1}}}
\@namedef{PY@tok@cm}{\let\PY@it=\textit\def\PY@tc##1{\textcolor[rgb]{0.25,0.50,0.50}{##1}}}
\@namedef{PY@tok@cpf}{\let\PY@it=\textit\def\PY@tc##1{\textcolor[rgb]{0.25,0.50,0.50}{##1}}}
\@namedef{PY@tok@c1}{\let\PY@it=\textit\def\PY@tc##1{\textcolor[rgb]{0.25,0.50,0.50}{##1}}}
\@namedef{PY@tok@cs}{\let\PY@it=\textit\def\PY@tc##1{\textcolor[rgb]{0.25,0.50,0.50}{##1}}}

\def\PYZbs{\char`\\}
\def\PYZus{\char`\_}
\def\PYZob{\char`\{}
\def\PYZcb{\char`\}}
\def\PYZca{\char`\^}
\def\PYZam{\char`\&}
\def\PYZlt{\char`\<}
\def\PYZgt{\char`\>}
\def\PYZsh{\char`\#}
\def\PYZpc{\char`\%}
\def\PYZdl{\char`\$}
\def\PYZhy{\char`\-}
\def\PYZsq{\char`\'}
\def\PYZdq{\char`\"}
\def\PYZti{\char`\~}
% for compatibility with earlier versions
\def\PYZat{@}
\def\PYZlb{[}
\def\PYZrb{]}
\makeatother


    % For linebreaks inside Verbatim environment from package fancyvrb. 
    \makeatletter
        \newbox\Wrappedcontinuationbox 
        \newbox\Wrappedvisiblespacebox 
        \newcommand*\Wrappedvisiblespace {\textcolor{red}{\textvisiblespace}} 
        \newcommand*\Wrappedcontinuationsymbol {\textcolor{red}{\llap{\tiny$\m@th\hookrightarrow$}}} 
        \newcommand*\Wrappedcontinuationindent {3ex } 
        \newcommand*\Wrappedafterbreak {\kern\Wrappedcontinuationindent\copy\Wrappedcontinuationbox} 
        % Take advantage of the already applied Pygments mark-up to insert 
        % potential linebreaks for TeX processing. 
        %        {, <, #, %, $, ' and ": go to next line. 
        %        _, }, ^, &, >, - and ~: stay at end of broken line. 
        % Use of \textquotesingle for straight quote. 
        \newcommand*\Wrappedbreaksatspecials {% 
            \def\PYGZus{\discretionary{\char`\_}{\Wrappedafterbreak}{\char`\_}}% 
            \def\PYGZob{\discretionary{}{\Wrappedafterbreak\char`\{}{\char`\{}}% 
            \def\PYGZcb{\discretionary{\char`\}}{\Wrappedafterbreak}{\char`\}}}% 
            \def\PYGZca{\discretionary{\char`\^}{\Wrappedafterbreak}{\char`\^}}% 
            \def\PYGZam{\discretionary{\char`\&}{\Wrappedafterbreak}{\char`\&}}% 
            \def\PYGZlt{\discretionary{}{\Wrappedafterbreak\char`\<}{\char`\<}}% 
            \def\PYGZgt{\discretionary{\char`\>}{\Wrappedafterbreak}{\char`\>}}% 
            \def\PYGZsh{\discretionary{}{\Wrappedafterbreak\char`\#}{\char`\#}}% 
            \def\PYGZpc{\discretionary{}{\Wrappedafterbreak\char`\%}{\char`\%}}% 
            \def\PYGZdl{\discretionary{}{\Wrappedafterbreak\char`\$}{\char`\$}}% 
            \def\PYGZhy{\discretionary{\char`\-}{\Wrappedafterbreak}{\char`\-}}% 
            \def\PYGZsq{\discretionary{}{\Wrappedafterbreak\textquotesingle}{\textquotesingle}}% 
            \def\PYGZdq{\discretionary{}{\Wrappedafterbreak\char`\"}{\char`\"}}% 
            \def\PYGZti{\discretionary{\char`\~}{\Wrappedafterbreak}{\char`\~}}% 
        } 
        % Some characters . , ; ? ! / are not pygmentized. 
        % This macro makes them "active" and they will insert potential linebreaks 
        \newcommand*\Wrappedbreaksatpunct {% 
            \lccode`\~`\.\lowercase{\def~}{\discretionary{\hbox{\char`\.}}{\Wrappedafterbreak}{\hbox{\char`\.}}}% 
            \lccode`\~`\,\lowercase{\def~}{\discretionary{\hbox{\char`\,}}{\Wrappedafterbreak}{\hbox{\char`\,}}}% 
            \lccode`\~`\;\lowercase{\def~}{\discretionary{\hbox{\char`\;}}{\Wrappedafterbreak}{\hbox{\char`\;}}}% 
            \lccode`\~`\:\lowercase{\def~}{\discretionary{\hbox{\char`\:}}{\Wrappedafterbreak}{\hbox{\char`\:}}}% 
            \lccode`\~`\?\lowercase{\def~}{\discretionary{\hbox{\char`\?}}{\Wrappedafterbreak}{\hbox{\char`\?}}}% 
            \lccode`\~`\!\lowercase{\def~}{\discretionary{\hbox{\char`\!}}{\Wrappedafterbreak}{\hbox{\char`\!}}}% 
            \lccode`\~`\/\lowercase{\def~}{\discretionary{\hbox{\char`\/}}{\Wrappedafterbreak}{\hbox{\char`\/}}}% 
            \catcode`\.\active
            \catcode`\,\active 
            \catcode`\;\active
            \catcode`\:\active
            \catcode`\?\active
            \catcode`\!\active
            \catcode`\/\active 
            \lccode`\~`\~ 	
        }
    \makeatother

    \let\OriginalVerbatim=\Verbatim
    \makeatletter
    \renewcommand{\Verbatim}[1][1]{%
        %\parskip\z@skip
        \sbox\Wrappedcontinuationbox {\Wrappedcontinuationsymbol}%
        \sbox\Wrappedvisiblespacebox {\FV@SetupFont\Wrappedvisiblespace}%
        \def\FancyVerbFormatLine ##1{\hsize\linewidth
            \vtop{\raggedright\hyphenpenalty\z@\exhyphenpenalty\z@
                \doublehyphendemerits\z@\finalhyphendemerits\z@
                \strut ##1\strut}%
        }%
        % If the linebreak is at a space, the latter will be displayed as visible
        % space at end of first line, and a continuation symbol starts next line.
        % Stretch/shrink are however usually zero for typewriter font.
        \def\FV@Space {%
            \nobreak\hskip\z@ plus\fontdimen3\font minus\fontdimen4\font
            \discretionary{\copy\Wrappedvisiblespacebox}{\Wrappedafterbreak}
            {\kern\fontdimen2\font}%
        }%
        
        % Allow breaks at special characters using \PYG... macros.
        \Wrappedbreaksatspecials
        % Breaks at punctuation characters . , ; ? ! and / need catcode=\active 	
        \OriginalVerbatim[#1,codes*=\Wrappedbreaksatpunct]%
    }
    \makeatother

    % Exact colors from NB
    \definecolor{incolor}{HTML}{303F9F}
    \definecolor{outcolor}{HTML}{D84315}
    \definecolor{cellborder}{HTML}{CFCFCF}
    \definecolor{cellbackground}{HTML}{F7F7F7}
    
    % prompt
    \makeatletter
    \newcommand{\boxspacing}{\kern\kvtcb@left@rule\kern\kvtcb@boxsep}
    \makeatother
    \newcommand{\prompt}[4]{
        {\ttfamily\llap{{\color{#2}[#3]:\hspace{3pt}#4}}\vspace{-\baselineskip}}
    }
    

    
    % Prevent overflowing lines due to hard-to-break entities
    \sloppy 
    % Setup hyperref package
    \hypersetup{
      breaklinks=true,  % so long urls are correctly broken across lines
      colorlinks=true,
      urlcolor=urlcolor,
      linkcolor=linkcolor,
      citecolor=citecolor,
      }
    % Slightly bigger margins than the latex defaults
    
    \geometry{verbose,tmargin=1in,bmargin=1in,lmargin=1in,rmargin=1in}
    
    

\begin{document}
    
    \maketitle
    
    

    
    \begin{tcolorbox}[breakable, size=fbox, boxrule=1pt, pad at break*=1mm,colback=cellbackground, colframe=cellborder]
\prompt{In}{incolor}{2}{\boxspacing}
\begin{Verbatim}[commandchars=\\\{\}]
\PY{k+kn}{import} \PY{n+nn}{pandas} \PY{k}{as} \PY{n+nn}{pd}
\PY{k+kn}{import} \PY{n+nn}{sklearn}
\PY{k+kn}{from} \PY{n+nn}{sklearn} \PY{k+kn}{import} \PY{n}{tree}
\PY{k+kn}{from} \PY{n+nn}{sklearn}\PY{n+nn}{.}\PY{n+nn}{metrics} \PY{k+kn}{import} \PY{n}{accuracy\PYZus{}score}
\PY{k+kn}{from} \PY{n+nn}{sklearn}\PY{n+nn}{.}\PY{n+nn}{model\PYZus{}selection} \PY{k+kn}{import} \PY{n}{train\PYZus{}test\PYZus{}split}
\PY{k+kn}{import} \PY{n+nn}{matplotlib}\PY{n+nn}{.}\PY{n+nn}{pyplot} \PY{k}{as} \PY{n+nn}{plt}
\end{Verbatim}
\end{tcolorbox}

    Reading data of file and classifying it in dependent and independent
variables (Y)

    \begin{tcolorbox}[breakable, size=fbox, boxrule=1pt, pad at break*=1mm,colback=cellbackground, colframe=cellborder]
\prompt{In}{incolor}{19}{\boxspacing}
\begin{Verbatim}[commandchars=\\\{\}]
\PY{n}{data} \PY{o}{=} \PY{n}{pd}\PY{o}{.}\PY{n}{read\PYZus{}csv}\PY{p}{(}\PY{l+s+s1}{\PYZsq{}}\PY{l+s+s1}{C:/Users/33789/Downloads/diabetes.csv}\PY{l+s+s1}{\PYZsq{}}\PY{p}{)}
\PY{n}{Y} \PY{o}{=} \PY{n}{data}\PY{p}{[}\PY{l+s+s1}{\PYZsq{}}\PY{l+s+s1}{class}\PY{l+s+s1}{\PYZsq{}}\PY{p}{]}
\PY{n}{X} \PY{o}{=} \PY{n}{data}\PY{o}{.}\PY{n}{drop}\PY{p}{(}\PY{p}{[}\PY{l+s+s1}{\PYZsq{}}\PY{l+s+s1}{class}\PY{l+s+s1}{\PYZsq{}}\PY{p}{]}\PY{p}{,}\PY{n}{axis}\PY{o}{=}\PY{l+m+mi}{1}\PY{p}{)}
\end{Verbatim}
\end{tcolorbox}

    Splitting data into training and testing

    \begin{tcolorbox}[breakable, size=fbox, boxrule=1pt, pad at break*=1mm,colback=cellbackground, colframe=cellborder]
\prompt{In}{incolor}{20}{\boxspacing}
\begin{Verbatim}[commandchars=\\\{\}]
\PY{n}{X\PYZus{}train}\PY{p}{,} \PY{n}{X\PYZus{}test}\PY{p}{,} \PY{n}{Y\PYZus{}train}\PY{p}{,} \PY{n}{Y\PYZus{}test} \PY{o}{=} \PY{n}{train\PYZus{}test\PYZus{}split}\PY{p}{(}\PY{n}{X}\PY{p}{,} \PY{n}{Y}\PY{p}{,} 
\PY{n}{test\PYZus{}size}\PY{o}{=}\PY{l+m+mf}{0.34}\PY{p}{,} \PY{n}{random\PYZus{}state}\PY{o}{=}\PY{l+m+mi}{10}\PY{p}{)}
\end{Verbatim}
\end{tcolorbox}

    Creating new decision tree, in thi scase with depth= 1

    clf = tree.DecisionTreeClassifier(criterion = `entropy', max\_depth=1)

    Fitting training data into our decision tree

    \begin{tcolorbox}[breakable, size=fbox, boxrule=1pt, pad at break*=1mm,colback=cellbackground, colframe=cellborder]
\prompt{In}{incolor}{26}{\boxspacing}
\begin{Verbatim}[commandchars=\\\{\}]
\PY{n}{clf} \PY{o}{=} \PY{n}{clf}\PY{o}{.}\PY{n}{fit}\PY{p}{(}\PY{n}{X\PYZus{}train}\PY{p}{,} \PY{n}{Y\PYZus{}train}\PY{p}{)}
\end{Verbatim}
\end{tcolorbox}

    Generating predictions of indep var from data

    \begin{tcolorbox}[breakable, size=fbox, boxrule=1pt, pad at break*=1mm,colback=cellbackground, colframe=cellborder]
\prompt{In}{incolor}{27}{\boxspacing}
\begin{Verbatim}[commandchars=\\\{\}]
\PY{n}{Ypredicted} \PY{o}{=} \PY{n}{clf}\PY{o}{.}\PY{n}{predict}\PY{p}{(}\PY{n}{X\PYZus{}train}\PY{p}{)}
\end{Verbatim}
\end{tcolorbox}

    Computing accuracy between the predictions to test data (we know this
one is true )

    \begin{tcolorbox}[breakable, size=fbox, boxrule=1pt, pad at break*=1mm,colback=cellbackground, colframe=cellborder]
\prompt{In}{incolor}{28}{\boxspacing}
\begin{Verbatim}[commandchars=\\\{\}]
\PY{n}{acc} \PY{o}{=} \PY{n}{accuracy\PYZus{}score}\PY{p}{(}\PY{n}{Y\PYZus{}train}\PY{p}{,} \PY{n}{Ypredicted}\PY{p}{)}
\end{Verbatim}
\end{tcolorbox}

    \begin{tcolorbox}[breakable, size=fbox, boxrule=1pt, pad at break*=1mm,colback=cellbackground, colframe=cellborder]
\prompt{In}{incolor}{29}{\boxspacing}
\begin{Verbatim}[commandchars=\\\{\}]
\PY{n+nb}{print}\PY{p}{(}\PY{n}{acc}\PY{p}{)}
\end{Verbatim}
\end{tcolorbox}

    \begin{Verbatim}[commandchars=\\\{\}]
0.7648221343873518
    \end{Verbatim}

    \begin{tcolorbox}[breakable, size=fbox, boxrule=1pt, pad at break*=1mm,colback=cellbackground, colframe=cellborder]
\prompt{In}{incolor}{31}{\boxspacing}
\begin{Verbatim}[commandchars=\\\{\}]
\PY{n}{Ypredicted\PYZus{}test} \PY{o}{=} \PY{n}{clf}\PY{o}{.}\PY{n}{predict}\PY{p}{(}\PY{n}{X\PYZus{}test}\PY{p}{)}
\end{Verbatim}
\end{tcolorbox}

    \begin{tcolorbox}[breakable, size=fbox, boxrule=1pt, pad at break*=1mm,colback=cellbackground, colframe=cellborder]
\prompt{In}{incolor}{32}{\boxspacing}
\begin{Verbatim}[commandchars=\\\{\}]
\PY{n}{acc} \PY{o}{=} \PY{n}{accuracy\PYZus{}score}\PY{p}{(}\PY{n}{Y\PYZus{}test}\PY{p}{,} \PY{n}{Ypredicted\PYZus{}test}\PY{p}{)}
\end{Verbatim}
\end{tcolorbox}

    \begin{tcolorbox}[breakable, size=fbox, boxrule=1pt, pad at break*=1mm,colback=cellbackground, colframe=cellborder]
\prompt{In}{incolor}{33}{\boxspacing}
\begin{Verbatim}[commandchars=\\\{\}]
\PY{n+nb}{print}\PY{p}{(}\PY{n}{acc}
     \PY{p}{)}
\end{Verbatim}
\end{tcolorbox}

    \begin{Verbatim}[commandchars=\\\{\}]
0.7213740458015268
    \end{Verbatim}

    As we can see the training accuracy is slighty higher than the testing
accuracy, since this one-depth tree was trained with training data and
thus having a greater accuracy

    \begin{tcolorbox}[breakable, size=fbox, boxrule=1pt, pad at break*=1mm,colback=cellbackground, colframe=cellborder]
\prompt{In}{incolor}{35}{\boxspacing}
\begin{Verbatim}[commandchars=\\\{\}]
\PY{n}{clf2} \PY{o}{=} \PY{n}{tree}\PY{o}{.}\PY{n}{DecisionTreeClassifier}\PY{p}{(}\PY{n}{criterion} \PY{o}{=} \PY{l+s+s1}{\PYZsq{}}\PY{l+s+s1}{entropy}\PY{l+s+s1}{\PYZsq{}}\PY{p}{,} \PY{n}{max\PYZus{}depth} \PY{o}{=} \PY{k+kc}{None}\PY{p}{)}
\end{Verbatim}
\end{tcolorbox}

    \begin{tcolorbox}[breakable, size=fbox, boxrule=1pt, pad at break*=1mm,colback=cellbackground, colframe=cellborder]
\prompt{In}{incolor}{36}{\boxspacing}
\begin{Verbatim}[commandchars=\\\{\}]
\PY{n}{clf2} \PY{o}{=} \PY{n}{clf2}\PY{o}{.}\PY{n}{fit}\PY{p}{(}\PY{n}{X\PYZus{}train}\PY{p}{,} \PY{n}{Y\PYZus{}train}\PY{p}{)}
\end{Verbatim}
\end{tcolorbox}

    \begin{tcolorbox}[breakable, size=fbox, boxrule=1pt, pad at break*=1mm,colback=cellbackground, colframe=cellborder]
\prompt{In}{incolor}{37}{\boxspacing}
\begin{Verbatim}[commandchars=\\\{\}]
\PY{n}{Ypredicted2} \PY{o}{=} \PY{n}{clf2}\PY{o}{.}\PY{n}{predict}\PY{p}{(}\PY{n}{X\PYZus{}train}\PY{p}{)}
\PY{n}{acc2} \PY{o}{=} \PY{n}{accuracy\PYZus{}score}\PY{p}{(}\PY{n}{Y\PYZus{}train}\PY{p}{,} \PY{n}{Ypredicted2}\PY{p}{)}
\PY{n+nb}{print}\PY{p}{(}\PY{n}{acc2}\PY{p}{)}
\end{Verbatim}
\end{tcolorbox}

    \begin{Verbatim}[commandchars=\\\{\}]
1.0
    \end{Verbatim}

    \begin{tcolorbox}[breakable, size=fbox, boxrule=1pt, pad at break*=1mm,colback=cellbackground, colframe=cellborder]
\prompt{In}{incolor}{39}{\boxspacing}
\begin{Verbatim}[commandchars=\\\{\}]
\PY{n}{Ypredicted\PYZus{}test2} \PY{o}{=} \PY{n}{clf2}\PY{o}{.}\PY{n}{predict}\PY{p}{(}\PY{n}{X\PYZus{}test}\PY{p}{)}
\PY{n}{acc} \PY{o}{=} \PY{n}{accuracy\PYZus{}score}\PY{p}{(}\PY{n}{Y\PYZus{}test}\PY{p}{,} \PY{n}{Ypredicted\PYZus{}test2}\PY{p}{)}
\PY{n+nb}{print}\PY{p}{(}\PY{n}{acc}\PY{p}{)}
\end{Verbatim}
\end{tcolorbox}

    \begin{Verbatim}[commandchars=\\\{\}]
0.7442748091603053
    \end{Verbatim}

    As we can see the accuracy for the training data is 1, this makes sense
since the decision tree has no limit on depths and can be modelled
exactly as the data provided, compared to one-depth tree that had 0.76
for the training accuracy since it has a boundary of 1 ply for the
entire tree, making it much more inaccurate. As in for the testing
accuracy the multiple-depths tree exceeds the one-depth but not for too
much having 0.74 and 0.72 respectively. The later is explained since the
multi-layered tree is expected to be modelled from the data provided and
teh other one is modelled by the data given that there is a maximum of
one level. Nevertheless, I was expecting this difference in
teting-accuracy be much more significant

    \begin{tcolorbox}[breakable, size=fbox, boxrule=1pt, pad at break*=1mm,colback=cellbackground, colframe=cellborder]
\prompt{In}{incolor}{41}{\boxspacing}
\begin{Verbatim}[commandchars=\\\{\}]
\PY{n}{clf} \PY{o}{=} \PY{n}{tree}\PY{o}{.}\PY{n}{DecisionTreeClassifier}\PY{p}{(}\PY{n}{criterion} \PY{o}{=} \PY{l+s+s1}{\PYZsq{}}\PY{l+s+s1}{entropy}\PY{l+s+s1}{\PYZsq{}}\PY{p}{,} \PY{n}{min\PYZus{}samples\PYZus{}leaf} \PY{o}{=} \PY{l+m+mi}{768}\PY{p}{)}
\end{Verbatim}
\end{tcolorbox}

    given that there is 768 instances, put the size on min\_samples\_leafs

    \begin{tcolorbox}[breakable, size=fbox, boxrule=1pt, pad at break*=1mm,colback=cellbackground, colframe=cellborder]
\prompt{In}{incolor}{43}{\boxspacing}
\begin{Verbatim}[commandchars=\\\{\}]
\PY{n}{clf} \PY{o}{=} \PY{n}{clf}\PY{o}{.}\PY{n}{fit}\PY{p}{(}\PY{n}{X\PYZus{}train}\PY{p}{,} \PY{n}{Y\PYZus{}train}\PY{p}{)}
\PY{n}{Ypredicted} \PY{o}{=} \PY{n}{clf}\PY{o}{.}\PY{n}{predict}\PY{p}{(}\PY{n}{X\PYZus{}train}\PY{p}{)}
\PY{n}{acc} \PY{o}{=} \PY{n}{accuracy\PYZus{}score}\PY{p}{(}\PY{n}{Y\PYZus{}train}\PY{p}{,} \PY{n}{Ypredicted}\PY{p}{)}
\PY{n+nb}{print}\PY{p}{(}\PY{l+s+s1}{\PYZsq{}}\PY{l+s+s1}{traing data acc}\PY{l+s+s1}{\PYZsq{}}\PY{p}{)}
\PY{n+nb}{print}\PY{p}{(}\PY{n}{acc} \PY{p}{)}
\PY{n}{Ypredicted\PYZus{}test} \PY{o}{=} \PY{n}{clf}\PY{o}{.}\PY{n}{predict}\PY{p}{(}\PY{n}{X\PYZus{}test}\PY{p}{)}
\PY{n}{acc2} \PY{o}{=} \PY{n}{accuracy\PYZus{}score}\PY{p}{(}\PY{n}{Y\PYZus{}test}\PY{p}{,} \PY{n}{Ypredicted\PYZus{}test}\PY{p}{)}
\PY{n+nb}{print}\PY{p}{(}\PY{l+s+s1}{\PYZsq{}}\PY{l+s+s1}{testing data acc}\PY{l+s+s1}{\PYZsq{}}\PY{p}{)}
\PY{n+nb}{print}\PY{p}{(}\PY{n}{acc2} \PY{p}{)}
\end{Verbatim}
\end{tcolorbox}

    \begin{Verbatim}[commandchars=\\\{\}]
traing data acc
0.6561264822134387
testing data acc
0.6412213740458015
    \end{Verbatim}

    Comparing all accuracies obtained, this later one is the smallest yet.
This might be because by instantating a tree that has a minimum number
of leafs of the size of the dataset it obliges it to have all nodes as
leafs, making it a one-depth tree, making the tree solely of the root
and then all the leaf nodes that in the multi\_level tree would
correspond to different plys where the classification fo the instances
help a better model and hence a higher accuracy

    \begin{tcolorbox}[breakable, size=fbox, boxrule=1pt, pad at break*=1mm,colback=cellbackground, colframe=cellborder]
\prompt{In}{incolor}{56}{\boxspacing}
\begin{Verbatim}[commandchars=\\\{\}]
\PY{n}{accuracy} \PY{o}{=} \PY{p}{[}\PY{p}{]}
\PY{n}{size\PYZus{}of\PYZus{}tree} \PY{o}{=} \PY{p}{[}\PY{p}{]}
\PY{k}{for} \PY{n}{x} \PY{o+ow}{in} \PY{n+nb}{range}\PY{p}{(}\PY{l+m+mi}{1}\PY{p}{,} \PY{l+m+mi}{769}\PY{p}{,} \PY{l+m+mi}{5}\PY{p}{)}\PY{p}{:}
    \PY{n}{clf} \PY{o}{=} \PY{n}{tree}\PY{o}{.}\PY{n}{DecisionTreeClassifier}\PY{p}{(}\PY{n}{criterion} \PY{o}{=} \PY{l+s+s1}{\PYZsq{}}\PY{l+s+s1}{entropy}\PY{l+s+s1}{\PYZsq{}}\PY{p}{,} \PY{n}{min\PYZus{}samples\PYZus{}leaf} \PY{o}{=} \PY{n}{x}\PY{p}{,} \PY{n}{max\PYZus{}depth} \PY{o}{=} \PY{k+kc}{None}\PY{p}{)}
    \PY{n}{size\PYZus{}of\PYZus{}tree}\PY{o}{.}\PY{n}{append}\PY{p}{(}\PY{n}{x}\PY{p}{)}
    \PY{n}{clf} \PY{o}{=} \PY{n}{clf}\PY{o}{.}\PY{n}{fit}\PY{p}{(}\PY{n}{X\PYZus{}train}\PY{p}{,} \PY{n}{Y\PYZus{}train}\PY{p}{)}
    \PY{n}{Ypredicted\PYZus{}test} \PY{o}{=} \PY{n}{clf}\PY{o}{.}\PY{n}{predict}\PY{p}{(}\PY{n}{X\PYZus{}test}\PY{p}{)}
    \PY{n}{acc} \PY{o}{=} \PY{n}{accuracy\PYZus{}score}\PY{p}{(}\PY{n}{Y\PYZus{}test}\PY{p}{,} \PY{n}{Ypredicted\PYZus{}test}\PY{p}{)}
    \PY{n}{accuracy}\PY{o}{.}\PY{n}{append}\PY{p}{(}\PY{n}{acc}\PY{p}{)}

\PY{n}{plt}\PY{o}{.}\PY{n}{plot}\PY{p}{(}\PY{n}{size\PYZus{}of\PYZus{}tree}\PY{p}{,} \PY{n}{accuracy}\PY{p}{)}
\PY{n}{plt}\PY{o}{.}\PY{n}{xlabel}\PY{p}{(}\PY{l+s+s1}{\PYZsq{}}\PY{l+s+s1}{Minimum number of sample leafs }\PY{l+s+s1}{\PYZsq{}}\PY{p}{)}
\PY{n}{plt}\PY{o}{.}\PY{n}{ylabel}\PY{p}{(}\PY{l+s+s1}{\PYZsq{}}\PY{l+s+s1}{Accuray of Decision tree }\PY{l+s+s1}{\PYZsq{}}\PY{p}{)}
    
\end{Verbatim}
\end{tcolorbox}

            \begin{tcolorbox}[breakable, size=fbox, boxrule=.5pt, pad at break*=1mm, opacityfill=0]
\prompt{Out}{outcolor}{56}{\boxspacing}
\begin{Verbatim}[commandchars=\\\{\}]
Text(0, 0.5, 'Accuray of Decision tree ')
\end{Verbatim}
\end{tcolorbox}
        
    \begin{center}
    \adjustimage{max size={0.9\linewidth}{0.9\paperheight}}{1ML_files/1ML_27_1.png}
    \end{center}
    { \hspace*{\fill} \\}
    
    *the range is until 769 since this last number won't be tested thus
ensuring 768 will be tested if it falls in the steps of 5

    \begin{tcolorbox}[breakable, size=fbox, boxrule=1pt, pad at break*=1mm,colback=cellbackground, colframe=cellborder]
\prompt{In}{incolor}{57}{\boxspacing}
\begin{Verbatim}[commandchars=\\\{\}]
\PY{n}{train} \PY{o}{=} \PY{p}{[}\PY{p}{]}
\PY{n}{test} \PY{o}{=} \PY{p}{[}\PY{p}{]}
\PY{n}{size\PYZus{}of\PYZus{}tree} \PY{o}{=} \PY{p}{[}\PY{p}{]}

\PY{k}{for} \PY{n}{x} \PY{o+ow}{in} \PY{n+nb}{range}\PY{p}{(}\PY{l+m+mi}{1}\PY{p}{,} \PY{l+m+mi}{769}\PY{p}{,} \PY{l+m+mi}{5}\PY{p}{)}\PY{p}{:}
    \PY{n}{clf} \PY{o}{=} \PY{n}{tree}\PY{o}{.}\PY{n}{DecisionTreeClassifier}\PY{p}{(}\PY{n}{criterion} \PY{o}{=} \PY{l+s+s1}{\PYZsq{}}\PY{l+s+s1}{entropy}\PY{l+s+s1}{\PYZsq{}}\PY{p}{,} \PY{n}{min\PYZus{}samples\PYZus{}leaf} \PY{o}{=} \PY{n}{x}\PY{p}{,} \PY{n}{max\PYZus{}depth} \PY{o}{=} \PY{k+kc}{None}\PY{p}{)}
    \PY{n}{size\PYZus{}of\PYZus{}tree}\PY{o}{.}\PY{n}{append}\PY{p}{(}\PY{n}{x}\PY{p}{)}
    \PY{n}{clf} \PY{o}{=} \PY{n}{clf}\PY{o}{.}\PY{n}{fit}\PY{p}{(}\PY{n}{X\PYZus{}train}\PY{p}{,} \PY{n}{Y\PYZus{}train}\PY{p}{)}
    \PY{n}{Ypredicted\PYZus{}train} \PY{o}{=} \PY{n}{clf}\PY{o}{.}\PY{n}{predict}\PY{p}{(}\PY{n}{X\PYZus{}train}\PY{p}{)}
    \PY{n}{Ypredicted\PYZus{}test} \PY{o}{=} \PY{n}{clf}\PY{o}{.}\PY{n}{predict}\PY{p}{(}\PY{n}{X\PYZus{}test}\PY{p}{)}
    \PY{n}{test}\PY{o}{.}\PY{n}{append}\PY{p}{(}\PY{n}{accuracy\PYZus{}score}\PY{p}{(}\PY{n}{Y\PYZus{}test}\PY{p}{,} \PY{n}{Ypredicted\PYZus{}test}\PY{p}{)}\PY{p}{)}
    \PY{n}{train}\PY{o}{.}\PY{n}{append}\PY{p}{(}\PY{n}{accuracy\PYZus{}score}\PY{p}{(}\PY{n}{Y\PYZus{}train}\PY{p}{,} \PY{n}{Ypredicted\PYZus{}train}\PY{p}{)}\PY{p}{)}

\PY{n}{plt}\PY{o}{.}\PY{n}{plot}\PY{p}{(}\PY{n}{train}\PY{p}{,} \PY{n}{label}\PY{o}{=} \PY{l+s+s1}{\PYZsq{}}\PY{l+s+s1}{train}\PY{l+s+s1}{\PYZsq{}}\PY{p}{)}
\PY{n}{plt}\PY{o}{.}\PY{n}{plot}\PY{p}{(}\PY{n}{test}\PY{p}{,} \PY{n}{label} \PY{o}{=} \PY{l+s+s1}{\PYZsq{}}\PY{l+s+s1}{test}\PY{l+s+s1}{\PYZsq{}}\PY{p}{)}
\PY{n}{plt}\PY{o}{.}\PY{n}{xlabel}\PY{p}{(}\PY{l+s+s1}{\PYZsq{}}\PY{l+s+s1}{Size of minimum sample leafs}\PY{l+s+s1}{\PYZsq{}}\PY{p}{)}
\PY{n}{plt}\PY{o}{.}\PY{n}{ylabel}\PY{p}{(}\PY{l+s+s1}{\PYZsq{}}\PY{l+s+s1}{Accuray of Decision tree }\PY{l+s+s1}{\PYZsq{}}\PY{p}{)}
\PY{n}{plt}\PY{o}{.}\PY{n}{legend}\PY{p}{(}\PY{n}{loc}\PY{o}{=}\PY{l+s+s1}{\PYZsq{}}\PY{l+s+s1}{upper center}\PY{l+s+s1}{\PYZsq{}}\PY{p}{)}
\end{Verbatim}
\end{tcolorbox}

            \begin{tcolorbox}[breakable, size=fbox, boxrule=.5pt, pad at break*=1mm, opacityfill=0]
\prompt{Out}{outcolor}{57}{\boxspacing}
\begin{Verbatim}[commandchars=\\\{\}]
<matplotlib.legend.Legend at 0x165b49a7df0>
\end{Verbatim}
\end{tcolorbox}
        
    \begin{center}
    \adjustimage{max size={0.9\linewidth}{0.9\paperheight}}{1ML_files/1ML_29_1.png}
    \end{center}
    { \hspace*{\fill} \\}
    
    In the previous plot, we can only see overfitting. The graph should be
read right to left, as the tree grows bigger, ie less minimum amount of
sample leafs ( since the rest of nodes will be internal nodes), the
difference between the accuracy of the training data and the testing
data grows apart. So when we reach the most left values the accuracy of
the training data is 1 (we saw this when we created a multileveled tree
) and the testing accuracy only reaches 0.75, thus a clear example of
overfitting the data. At the beginning, when both accuracies lines are
parallel, we can classify this as underfitting.

    \hypertarget{glass-dataset}{%
\section{glass dataset}\label{glass-dataset}}

    experiment in the same manner with different dataset, namely glass.csv

    \begin{tcolorbox}[breakable, size=fbox, boxrule=1pt, pad at break*=1mm,colback=cellbackground, colframe=cellborder]
\prompt{In}{incolor}{59}{\boxspacing}
\begin{Verbatim}[commandchars=\\\{\}]
\PY{n}{data} \PY{o}{=} \PY{n}{pd}\PY{o}{.}\PY{n}{read\PYZus{}csv}\PY{p}{(}\PY{l+s+s1}{\PYZsq{}}\PY{l+s+s1}{C:/Users/33789/Downloads/glass.csv}\PY{l+s+s1}{\PYZsq{}}\PY{p}{)}
\PY{n}{Y} \PY{o}{=} \PY{n}{data}\PY{p}{[}\PY{l+s+s1}{\PYZsq{}}\PY{l+s+s1}{class}\PY{l+s+s1}{\PYZsq{}}\PY{p}{]}
\PY{n}{X} \PY{o}{=} \PY{n}{data}\PY{o}{.}\PY{n}{drop}\PY{p}{(}\PY{p}{[}\PY{l+s+s1}{\PYZsq{}}\PY{l+s+s1}{class}\PY{l+s+s1}{\PYZsq{}}\PY{p}{]}\PY{p}{,}\PY{n}{axis}\PY{o}{=}\PY{l+m+mi}{1}\PY{p}{)}
\end{Verbatim}
\end{tcolorbox}

    \begin{tcolorbox}[breakable, size=fbox, boxrule=1pt, pad at break*=1mm,colback=cellbackground, colframe=cellborder]
\prompt{In}{incolor}{60}{\boxspacing}
\begin{Verbatim}[commandchars=\\\{\}]
\PY{n}{X\PYZus{}train}\PY{p}{,} \PY{n}{X\PYZus{}test}\PY{p}{,} \PY{n}{Y\PYZus{}train}\PY{p}{,} \PY{n}{Y\PYZus{}test} \PY{o}{=} \PY{n}{train\PYZus{}test\PYZus{}split}\PY{p}{(}\PY{n}{X}\PY{p}{,} \PY{n}{Y}\PY{p}{,} 
\PY{n}{test\PYZus{}size}\PY{o}{=}\PY{l+m+mf}{0.34}\PY{p}{,} \PY{n}{random\PYZus{}state}\PY{o}{=}\PY{l+m+mi}{10}\PY{p}{)}
\end{Verbatim}
\end{tcolorbox}

    creating one depth tree

    \begin{tcolorbox}[breakable, size=fbox, boxrule=1pt, pad at break*=1mm,colback=cellbackground, colframe=cellborder]
\prompt{In}{incolor}{61}{\boxspacing}
\begin{Verbatim}[commandchars=\\\{\}]
\PY{n}{clf} \PY{o}{=} \PY{n}{tree}\PY{o}{.}\PY{n}{DecisionTreeClassifier}\PY{p}{(}\PY{n}{criterion} \PY{o}{=} \PY{l+s+s1}{\PYZsq{}}\PY{l+s+s1}{entropy}\PY{l+s+s1}{\PYZsq{}}\PY{p}{,} \PY{n}{max\PYZus{}depth}\PY{o}{=}\PY{l+m+mi}{1}\PY{p}{)}
\PY{n}{clf} \PY{o}{=} \PY{n}{clf}\PY{o}{.}\PY{n}{fit}\PY{p}{(}\PY{n}{X\PYZus{}train}\PY{p}{,} \PY{n}{Y\PYZus{}train}\PY{p}{)}
\end{Verbatim}
\end{tcolorbox}

    \begin{tcolorbox}[breakable, size=fbox, boxrule=1pt, pad at break*=1mm,colback=cellbackground, colframe=cellborder]
\prompt{In}{incolor}{62}{\boxspacing}
\begin{Verbatim}[commandchars=\\\{\}]
\PY{n}{Ypredicted} \PY{o}{=} \PY{n}{clf}\PY{o}{.}\PY{n}{predict}\PY{p}{(}\PY{n}{X\PYZus{}train}\PY{p}{)}
\PY{n}{acc} \PY{o}{=} \PY{n}{accuracy\PYZus{}score}\PY{p}{(}\PY{n}{Y\PYZus{}train}\PY{p}{,} \PY{n}{Ypredicted}\PY{p}{)}
\PY{n+nb}{print} \PY{p}{(}\PY{n}{acc}\PY{p}{)}
\end{Verbatim}
\end{tcolorbox}

    \begin{Verbatim}[commandchars=\\\{\}]
0.46099290780141844
    \end{Verbatim}

    accuracy of testing data

    \begin{tcolorbox}[breakable, size=fbox, boxrule=1pt, pad at break*=1mm,colback=cellbackground, colframe=cellborder]
\prompt{In}{incolor}{67}{\boxspacing}
\begin{Verbatim}[commandchars=\\\{\}]
\PY{n}{Ypredicted\PYZus{}test2} \PY{o}{=} \PY{n}{clf}\PY{o}{.}\PY{n}{predict}\PY{p}{(}\PY{n}{X\PYZus{}test}\PY{p}{)}
\PY{n}{acc} \PY{o}{=} \PY{n}{accuracy\PYZus{}score}\PY{p}{(}\PY{n}{Y\PYZus{}test}\PY{p}{,} \PY{n}{Ypredicted\PYZus{}test2}\PY{p}{)}
\PY{n+nb}{print}\PY{p}{(}\PY{n}{acc}\PY{p}{)}
\end{Verbatim}
\end{tcolorbox}

    \begin{Verbatim}[commandchars=\\\{\}]
0.4246575342465753
    \end{Verbatim}

    Indeed test accuracy is lower than train accuracy

    Multi-level decision tree

    \begin{tcolorbox}[breakable, size=fbox, boxrule=1pt, pad at break*=1mm,colback=cellbackground, colframe=cellborder]
\prompt{In}{incolor}{69}{\boxspacing}
\begin{Verbatim}[commandchars=\\\{\}]
\PY{n}{clf2} \PY{o}{=} \PY{n}{tree}\PY{o}{.}\PY{n}{DecisionTreeClassifier}\PY{p}{(}\PY{n}{criterion} \PY{o}{=} \PY{l+s+s1}{\PYZsq{}}\PY{l+s+s1}{entropy}\PY{l+s+s1}{\PYZsq{}}\PY{p}{,} \PY{n}{max\PYZus{}depth} \PY{o}{=} \PY{k+kc}{None}\PY{p}{)}
\PY{n}{clf2} \PY{o}{=} \PY{n}{clf2}\PY{o}{.}\PY{n}{fit}\PY{p}{(}\PY{n}{X\PYZus{}train}\PY{p}{,} \PY{n}{Y\PYZus{}train}\PY{p}{)}
\end{Verbatim}
\end{tcolorbox}

    \begin{tcolorbox}[breakable, size=fbox, boxrule=1pt, pad at break*=1mm,colback=cellbackground, colframe=cellborder]
\prompt{In}{incolor}{70}{\boxspacing}
\begin{Verbatim}[commandchars=\\\{\}]
\PY{n}{Ypredicted2} \PY{o}{=} \PY{n}{clf2}\PY{o}{.}\PY{n}{predict}\PY{p}{(}\PY{n}{X\PYZus{}train}\PY{p}{)}
\PY{n}{acc2} \PY{o}{=} \PY{n}{accuracy\PYZus{}score}\PY{p}{(}\PY{n}{Y\PYZus{}train}\PY{p}{,} \PY{n}{Ypredicted2}\PY{p}{)}
\PY{n+nb}{print}\PY{p}{(}\PY{n}{acc2}\PY{p}{)}
\end{Verbatim}
\end{tcolorbox}

    \begin{Verbatim}[commandchars=\\\{\}]
1.0
    \end{Verbatim}

    \begin{tcolorbox}[breakable, size=fbox, boxrule=1pt, pad at break*=1mm,colback=cellbackground, colframe=cellborder]
\prompt{In}{incolor}{71}{\boxspacing}
\begin{Verbatim}[commandchars=\\\{\}]
\PY{n}{Ypredicted2\PYZus{}test} \PY{o}{=} \PY{n}{clf2}\PY{o}{.}\PY{n}{predict}\PY{p}{(}\PY{n}{X\PYZus{}test}\PY{p}{)}
\PY{n}{acc2} \PY{o}{=} \PY{n}{accuracy\PYZus{}score}\PY{p}{(}\PY{n}{Y\PYZus{}test}\PY{p}{,} \PY{n}{Ypredicted2\PYZus{}test}\PY{p}{)}
\PY{n+nb}{print}\PY{p}{(}\PY{n}{acc2}\PY{p}{)}
\end{Verbatim}
\end{tcolorbox}

    \begin{Verbatim}[commandchars=\\\{\}]
0.6301369863013698
    \end{Verbatim}

    Multilevel tree has a much higher accuracy than onelevel tree given that
thanks to the multiple levels it helps models better the data and thus
be more accurate

    Testing pre-pruning error by modifing the number of leafs the decision
tree has

    \begin{tcolorbox}[breakable, size=fbox, boxrule=1pt, pad at break*=1mm,colback=cellbackground, colframe=cellborder]
\prompt{In}{incolor}{73}{\boxspacing}
\begin{Verbatim}[commandchars=\\\{\}]
\PY{n}{clf} \PY{o}{=} \PY{n}{tree}\PY{o}{.}\PY{n}{DecisionTreeClassifier}\PY{p}{(}\PY{n}{criterion} \PY{o}{=} \PY{l+s+s1}{\PYZsq{}}\PY{l+s+s1}{entropy}\PY{l+s+s1}{\PYZsq{}}\PY{p}{,} \PY{n}{min\PYZus{}samples\PYZus{}leaf} \PY{o}{=} \PY{l+m+mi}{768}\PY{p}{)}
\end{Verbatim}
\end{tcolorbox}

    \begin{tcolorbox}[breakable, size=fbox, boxrule=1pt, pad at break*=1mm,colback=cellbackground, colframe=cellborder]
\prompt{In}{incolor}{74}{\boxspacing}
\begin{Verbatim}[commandchars=\\\{\}]
\PY{n}{clf} \PY{o}{=} \PY{n}{clf}\PY{o}{.}\PY{n}{fit}\PY{p}{(}\PY{n}{X\PYZus{}train}\PY{p}{,} \PY{n}{Y\PYZus{}train}\PY{p}{)}
\PY{n}{Ypredicted} \PY{o}{=} \PY{n}{clf}\PY{o}{.}\PY{n}{predict}\PY{p}{(}\PY{n}{X\PYZus{}train}\PY{p}{)}
\PY{n}{acc} \PY{o}{=} \PY{n}{accuracy\PYZus{}score}\PY{p}{(}\PY{n}{Y\PYZus{}train}\PY{p}{,} \PY{n}{Ypredicted}\PY{p}{)}
\PY{n+nb}{print}\PY{p}{(}\PY{n}{acc} \PY{p}{)}
\end{Verbatim}
\end{tcolorbox}

    \begin{Verbatim}[commandchars=\\\{\}]
0.3546099290780142
    \end{Verbatim}

    \begin{tcolorbox}[breakable, size=fbox, boxrule=1pt, pad at break*=1mm,colback=cellbackground, colframe=cellborder]
\prompt{In}{incolor}{75}{\boxspacing}
\begin{Verbatim}[commandchars=\\\{\}]
\PY{n}{Ypredicted\PYZus{}test} \PY{o}{=} \PY{n}{clf}\PY{o}{.}\PY{n}{predict}\PY{p}{(}\PY{n}{X\PYZus{}test}\PY{p}{)}
\PY{n}{acc2} \PY{o}{=} \PY{n}{accuracy\PYZus{}score}\PY{p}{(}\PY{n}{Y\PYZus{}test}\PY{p}{,} \PY{n}{Ypredicted\PYZus{}test}\PY{p}{)}
\PY{n+nb}{print}\PY{p}{(}\PY{l+s+s1}{\PYZsq{}}\PY{l+s+s1}{testing data acc}\PY{l+s+s1}{\PYZsq{}}\PY{p}{)}
\PY{n+nb}{print}\PY{p}{(}\PY{n}{acc2} \PY{p}{)}
\end{Verbatim}
\end{tcolorbox}

    \begin{Verbatim}[commandchars=\\\{\}]
testing data acc
0.3561643835616438
    \end{Verbatim}

    The difference in accuracy of testing and training are surprinsingly
small, this might be explained by the already relative low accuracy we
obtained at the beginning while using the glass-dataset, and now
computing the accuracy of a decision tree that has all nodes as leafs
but the root returns us a even smaller accuracy. (this makes sense since
this decision tree is not optimal for modelling the data and hence
obtaining an even smaller accuracy than before)

    \begin{tcolorbox}[breakable, size=fbox, boxrule=1pt, pad at break*=1mm,colback=cellbackground, colframe=cellborder]
\prompt{In}{incolor}{8}{\boxspacing}
\begin{Verbatim}[commandchars=\\\{\}]
\PY{n}{train} \PY{o}{=} \PY{p}{[}\PY{p}{]}
\PY{n}{test} \PY{o}{=} \PY{p}{[}\PY{p}{]}
\PY{n}{size\PYZus{}of\PYZus{}tree} \PY{o}{=} \PY{p}{[}\PY{p}{]}

\PY{k}{for} \PY{n}{x} \PY{o+ow}{in} \PY{n+nb}{range}\PY{p}{(}\PY{l+m+mi}{1}\PY{p}{,} \PY{l+m+mi}{215}\PY{p}{,} \PY{l+m+mi}{5}\PY{p}{)}\PY{p}{:}
    \PY{n}{clf} \PY{o}{=} \PY{n}{tree}\PY{o}{.}\PY{n}{DecisionTreeClassifier}\PY{p}{(}\PY{n}{criterion} \PY{o}{=} \PY{l+s+s1}{\PYZsq{}}\PY{l+s+s1}{entropy}\PY{l+s+s1}{\PYZsq{}}\PY{p}{,} \PY{n}{min\PYZus{}samples\PYZus{}leaf} \PY{o}{=} \PY{n}{x}\PY{p}{,} \PY{n}{max\PYZus{}depth} \PY{o}{=} \PY{k+kc}{None}\PY{p}{)}
    \PY{n}{size\PYZus{}of\PYZus{}tree}\PY{o}{.}\PY{n}{append}\PY{p}{(}\PY{n}{x}\PY{p}{)}
    \PY{n}{clf} \PY{o}{=} \PY{n}{clf}\PY{o}{.}\PY{n}{fit}\PY{p}{(}\PY{n}{X\PYZus{}train}\PY{p}{,} \PY{n}{Y\PYZus{}train}\PY{p}{)}
    \PY{n}{Ypredicted\PYZus{}train} \PY{o}{=} \PY{n}{clf}\PY{o}{.}\PY{n}{predict}\PY{p}{(}\PY{n}{X\PYZus{}train}\PY{p}{)}
    \PY{n}{Ypredicted\PYZus{}test} \PY{o}{=} \PY{n}{clf}\PY{o}{.}\PY{n}{predict}\PY{p}{(}\PY{n}{X\PYZus{}test}\PY{p}{)}
    \PY{n}{test}\PY{o}{.}\PY{n}{append}\PY{p}{(}\PY{n}{accuracy\PYZus{}score}\PY{p}{(}\PY{n}{Y\PYZus{}test}\PY{p}{,} \PY{n}{Ypredicted\PYZus{}test}\PY{p}{)}\PY{p}{)}
    \PY{n}{train}\PY{o}{.}\PY{n}{append}\PY{p}{(}\PY{n}{accuracy\PYZus{}score}\PY{p}{(}\PY{n}{Y\PYZus{}train}\PY{p}{,} \PY{n}{Ypredicted\PYZus{}train}\PY{p}{)}\PY{p}{)}

\PY{n}{plt}\PY{o}{.}\PY{n}{plot}\PY{p}{(}\PY{n}{train}\PY{p}{,} \PY{n}{label}\PY{o}{=} \PY{l+s+s1}{\PYZsq{}}\PY{l+s+s1}{train}\PY{l+s+s1}{\PYZsq{}}\PY{p}{)}
\PY{n}{plt}\PY{o}{.}\PY{n}{plot}\PY{p}{(}\PY{n}{test}\PY{p}{,} \PY{n}{label} \PY{o}{=} \PY{l+s+s1}{\PYZsq{}}\PY{l+s+s1}{test}\PY{l+s+s1}{\PYZsq{}}\PY{p}{)}
\PY{n}{plt}\PY{o}{.}\PY{n}{xlabel}\PY{p}{(}\PY{l+s+s1}{\PYZsq{}}\PY{l+s+s1}{Size of minimum sample leafs}\PY{l+s+s1}{\PYZsq{}}\PY{p}{)}
\PY{n}{plt}\PY{o}{.}\PY{n}{ylabel}\PY{p}{(}\PY{l+s+s1}{\PYZsq{}}\PY{l+s+s1}{Accuray of Decision tree }\PY{l+s+s1}{\PYZsq{}}\PY{p}{)}
\PY{n}{plt}\PY{o}{.}\PY{n}{legend}\PY{p}{(}\PY{n}{loc}\PY{o}{=}\PY{l+s+s1}{\PYZsq{}}\PY{l+s+s1}{upper center}\PY{l+s+s1}{\PYZsq{}}\PY{p}{)}
\end{Verbatim}
\end{tcolorbox}

            \begin{tcolorbox}[breakable, size=fbox, boxrule=.5pt, pad at break*=1mm, opacityfill=0]
\prompt{Out}{outcolor}{8}{\boxspacing}
\begin{Verbatim}[commandchars=\\\{\}]
<matplotlib.legend.Legend at 0x224afc2a6d0>
\end{Verbatim}
\end{tcolorbox}
        
    \begin{center}
    \adjustimage{max size={0.9\linewidth}{0.9\paperheight}}{1ML_files/1ML_51_1.png}
    \end{center}
    { \hspace*{\fill} \\}
    
    At the left side of the graph the data is underfitted (the two lines are
parallel) ,just when the two lines get really close we could consider it
optimal and when the accuracy of the training increase at a higher rate
than the testing line it is considered as overfitting

    \hypertarget{testing-a-bit-more}{%
\section{testing a bit more}\label{testing-a-bit-more}}

    \begin{tcolorbox}[breakable, size=fbox, boxrule=1pt, pad at break*=1mm,colback=cellbackground, colframe=cellborder]
\prompt{In}{incolor}{3}{\boxspacing}
\begin{Verbatim}[commandchars=\\\{\}]
\PY{n}{data} \PY{o}{=} \PY{n}{pd}\PY{o}{.}\PY{n}{read\PYZus{}csv}\PY{p}{(}\PY{l+s+s1}{\PYZsq{}}\PY{l+s+s1}{C:/Users/33789/Downloads/glass.csv}\PY{l+s+s1}{\PYZsq{}}\PY{p}{)}
\PY{n}{Y} \PY{o}{=} \PY{n}{data}\PY{p}{[}\PY{l+s+s1}{\PYZsq{}}\PY{l+s+s1}{class}\PY{l+s+s1}{\PYZsq{}}\PY{p}{]}
\PY{n}{X} \PY{o}{=} \PY{n}{data}\PY{o}{.}\PY{n}{drop}\PY{p}{(}\PY{p}{[}\PY{l+s+s1}{\PYZsq{}}\PY{l+s+s1}{class}\PY{l+s+s1}{\PYZsq{}}\PY{p}{]}\PY{p}{,}\PY{n}{axis}\PY{o}{=}\PY{l+m+mi}{1}\PY{p}{)}
\end{Verbatim}
\end{tcolorbox}

    \begin{tcolorbox}[breakable, size=fbox, boxrule=1pt, pad at break*=1mm,colback=cellbackground, colframe=cellborder]
\prompt{In}{incolor}{4}{\boxspacing}
\begin{Verbatim}[commandchars=\\\{\}]
\PY{n}{X\PYZus{}train}\PY{p}{,} \PY{n}{X\PYZus{}test}\PY{p}{,} \PY{n}{Y\PYZus{}train}\PY{p}{,} \PY{n}{Y\PYZus{}test} \PY{o}{=} \PY{n}{train\PYZus{}test\PYZus{}split}\PY{p}{(}\PY{n}{X}\PY{p}{,} \PY{n}{Y}\PY{p}{,} 
\PY{n}{test\PYZus{}size}\PY{o}{=}\PY{l+m+mf}{0.20}\PY{p}{,} \PY{n}{random\PYZus{}state}\PY{o}{=}\PY{l+m+mi}{10}\PY{p}{)}
\end{Verbatim}
\end{tcolorbox}

    \begin{tcolorbox}[breakable, size=fbox, boxrule=1pt, pad at break*=1mm,colback=cellbackground, colframe=cellborder]
\prompt{In}{incolor}{5}{\boxspacing}
\begin{Verbatim}[commandchars=\\\{\}]
\PY{n}{clf} \PY{o}{=} \PY{n}{tree}\PY{o}{.}\PY{n}{DecisionTreeClassifier}\PY{p}{(}\PY{n}{criterion} \PY{o}{=} \PY{l+s+s1}{\PYZsq{}}\PY{l+s+s1}{entropy}\PY{l+s+s1}{\PYZsq{}}\PY{p}{,} \PY{n}{max\PYZus{}depth}\PY{o}{=}\PY{l+m+mi}{1}\PY{p}{)}
\PY{n}{clf} \PY{o}{=} \PY{n}{clf}\PY{o}{.}\PY{n}{fit}\PY{p}{(}\PY{n}{X\PYZus{}train}\PY{p}{,} \PY{n}{Y\PYZus{}train}\PY{p}{)}
\PY{n}{Ypredicted} \PY{o}{=} \PY{n}{clf}\PY{o}{.}\PY{n}{predict}\PY{p}{(}\PY{n}{X\PYZus{}train}\PY{p}{)}
\PY{n}{acc} \PY{o}{=} \PY{n}{accuracy\PYZus{}score}\PY{p}{(}\PY{n}{Y\PYZus{}train}\PY{p}{,} \PY{n}{Ypredicted}\PY{p}{)}
\PY{n+nb}{print} \PY{p}{(}\PY{n}{acc}\PY{p}{)}
\end{Verbatim}
\end{tcolorbox}

    \begin{Verbatim}[commandchars=\\\{\}]
0.4327485380116959
    \end{Verbatim}

    \begin{tcolorbox}[breakable, size=fbox, boxrule=1pt, pad at break*=1mm,colback=cellbackground, colframe=cellborder]
\prompt{In}{incolor}{6}{\boxspacing}
\begin{Verbatim}[commandchars=\\\{\}]
\PY{n}{Ypredicted\PYZus{}test2} \PY{o}{=} \PY{n}{clf}\PY{o}{.}\PY{n}{predict}\PY{p}{(}\PY{n}{X\PYZus{}test}\PY{p}{)}
\PY{n}{acc} \PY{o}{=} \PY{n}{accuracy\PYZus{}score}\PY{p}{(}\PY{n}{Y\PYZus{}test}\PY{p}{,} \PY{n}{Ypredicted\PYZus{}test2}\PY{p}{)}
\PY{n+nb}{print}\PY{p}{(}\PY{n}{acc}\PY{p}{)}
\end{Verbatim}
\end{tcolorbox}

    \begin{Verbatim}[commandchars=\\\{\}]
0.5116279069767442
    \end{Verbatim}

    By reducing the size of the test set we obtained a higher accuracy, this
can be explained since there is more data used to build the model thus
achieving a better structured or better informed decision tree and thus
obtaining a higher accuracy than when the test set was 0.34 approx 1/3
of the entire data set, in this case the test set is 1/5 of the entire
dataset. Something very weird is presented here, the testing accuracy is
bigger than the training accuracy!! this is very interesting :)

    \begin{tcolorbox}[breakable, size=fbox, boxrule=1pt, pad at break*=1mm,colback=cellbackground, colframe=cellborder]
\prompt{In}{incolor}{7}{\boxspacing}
\begin{Verbatim}[commandchars=\\\{\}]
\PY{n}{X\PYZus{}train}\PY{p}{,} \PY{n}{X\PYZus{}test}\PY{p}{,} \PY{n}{Y\PYZus{}train}\PY{p}{,} \PY{n}{Y\PYZus{}test} \PY{o}{=} \PY{n}{train\PYZus{}test\PYZus{}split}\PY{p}{(}\PY{n}{X}\PY{p}{,} \PY{n}{Y}\PY{p}{,} 
\PY{n}{test\PYZus{}size}\PY{o}{=}\PY{l+m+mf}{0.40}\PY{p}{,} \PY{n}{random\PYZus{}state}\PY{o}{=}\PY{l+m+mi}{10}\PY{p}{)}
\PY{n}{clf} \PY{o}{=} \PY{n}{tree}\PY{o}{.}\PY{n}{DecisionTreeClassifier}\PY{p}{(}\PY{n}{criterion} \PY{o}{=} \PY{l+s+s1}{\PYZsq{}}\PY{l+s+s1}{entropy}\PY{l+s+s1}{\PYZsq{}}\PY{p}{,} \PY{n}{max\PYZus{}depth}\PY{o}{=}\PY{l+m+mi}{1}\PY{p}{)}
\PY{n}{clf} \PY{o}{=} \PY{n}{clf}\PY{o}{.}\PY{n}{fit}\PY{p}{(}\PY{n}{X\PYZus{}train}\PY{p}{,} \PY{n}{Y\PYZus{}train}\PY{p}{)}
\PY{n}{Ypredicted} \PY{o}{=} \PY{n}{clf}\PY{o}{.}\PY{n}{predict}\PY{p}{(}\PY{n}{X\PYZus{}train}\PY{p}{)}
\PY{n}{acc} \PY{o}{=} \PY{n}{accuracy\PYZus{}score}\PY{p}{(}\PY{n}{Y\PYZus{}train}\PY{p}{,} \PY{n}{Ypredicted}\PY{p}{)}
\PY{n+nb}{print} \PY{p}{(}\PY{l+s+s1}{\PYZsq{}}\PY{l+s+s1}{ training acc}\PY{l+s+s1}{\PYZsq{}}\PY{p}{)}
\PY{n+nb}{print} \PY{p}{(}\PY{n}{acc}\PY{p}{)}
\PY{n}{Ypredicted\PYZus{}test2} \PY{o}{=} \PY{n}{clf}\PY{o}{.}\PY{n}{predict}\PY{p}{(}\PY{n}{X\PYZus{}test}\PY{p}{)}
\PY{n}{acc2} \PY{o}{=} \PY{n}{accuracy\PYZus{}score}\PY{p}{(}\PY{n}{Y\PYZus{}test}\PY{p}{,} \PY{n}{Ypredicted\PYZus{}test2}\PY{p}{)}
\PY{n+nb}{print} \PY{p}{(}\PY{l+s+s1}{\PYZsq{}}\PY{l+s+s1}{testing acc}\PY{l+s+s1}{\PYZsq{}}\PY{p}{)}
\PY{n+nb}{print}\PY{p}{(}\PY{n}{acc2}\PY{p}{)}
\end{Verbatim}
\end{tcolorbox}

    \begin{Verbatim}[commandchars=\\\{\}]
 training acc
0.46875
testing acc
0.4186046511627907
    \end{Verbatim}

    Now we reduced the training set to 3/5 and test set to 2/5, as a result
we come back to our normal proportion such that the training acc is
higher than the testing acc and we obtained similar accuracies to the
original split of the data

    \begin{tcolorbox}[breakable, size=fbox, boxrule=1pt, pad at break*=1mm,colback=cellbackground, colframe=cellborder]
\prompt{In}{incolor}{13}{\boxspacing}
\begin{Verbatim}[commandchars=\\\{\}]
\PY{n}{train} \PY{o}{=} \PY{p}{[}\PY{p}{]}
\PY{n}{test} \PY{o}{=} \PY{p}{[}\PY{p}{]}
\PY{n}{test\PYZus{}size} \PY{o}{=} \PY{p}{[}\PY{p}{]}

\PY{k}{for} \PY{n}{x} \PY{o+ow}{in} \PY{n+nb}{range}\PY{p}{(}\PY{l+m+mi}{1}\PY{p}{,} \PY{l+m+mi}{100} \PY{p}{,} \PY{l+m+mi}{1}\PY{p}{)}\PY{p}{:}
    \PY{n}{y} \PY{o}{=} \PY{n}{x}\PY{o}{/}\PY{l+m+mi}{100}
    \PY{n}{X\PYZus{}train}\PY{p}{,} \PY{n}{X\PYZus{}test}\PY{p}{,} \PY{n}{Y\PYZus{}train}\PY{p}{,} \PY{n}{Y\PYZus{}test} \PY{o}{=} \PY{n}{train\PYZus{}test\PYZus{}split}\PY{p}{(}\PY{n}{X}\PY{p}{,} \PY{n}{Y}\PY{p}{,} 
\PY{n}{test\PYZus{}size}\PY{o}{=} \PY{n}{y}\PY{p}{,} \PY{n}{random\PYZus{}state}\PY{o}{=}\PY{l+m+mi}{10}\PY{p}{)}
    \PY{n}{clf} \PY{o}{=} \PY{n}{tree}\PY{o}{.}\PY{n}{DecisionTreeClassifier}\PY{p}{(}\PY{n}{criterion} \PY{o}{=} \PY{l+s+s1}{\PYZsq{}}\PY{l+s+s1}{entropy}\PY{l+s+s1}{\PYZsq{}}\PY{p}{,} \PY{n}{max\PYZus{}depth} \PY{o}{=} \PY{k+kc}{None}\PY{p}{)}
    \PY{n}{test\PYZus{}size}\PY{o}{.}\PY{n}{append}\PY{p}{(}\PY{n}{y}\PY{p}{)}
    \PY{n}{clf} \PY{o}{=} \PY{n}{clf}\PY{o}{.}\PY{n}{fit}\PY{p}{(}\PY{n}{X\PYZus{}train}\PY{p}{,} \PY{n}{Y\PYZus{}train}\PY{p}{)}
    \PY{n}{Ypredicted\PYZus{}train} \PY{o}{=} \PY{n}{clf}\PY{o}{.}\PY{n}{predict}\PY{p}{(}\PY{n}{X\PYZus{}train}\PY{p}{)}
    \PY{n}{Ypredicted\PYZus{}test} \PY{o}{=} \PY{n}{clf}\PY{o}{.}\PY{n}{predict}\PY{p}{(}\PY{n}{X\PYZus{}test}\PY{p}{)}
    \PY{n}{test}\PY{o}{.}\PY{n}{append}\PY{p}{(}\PY{n}{accuracy\PYZus{}score}\PY{p}{(}\PY{n}{Y\PYZus{}test}\PY{p}{,} \PY{n}{Ypredicted\PYZus{}test}\PY{p}{)}\PY{p}{)}
    \PY{n}{train}\PY{o}{.}\PY{n}{append}\PY{p}{(}\PY{n}{accuracy\PYZus{}score}\PY{p}{(}\PY{n}{Y\PYZus{}train}\PY{p}{,} \PY{n}{Ypredicted\PYZus{}train}\PY{p}{)}\PY{p}{)}

\PY{n}{plt}\PY{o}{.}\PY{n}{plot}\PY{p}{(}\PY{n}{train}\PY{p}{,} \PY{n}{label}\PY{o}{=} \PY{l+s+s1}{\PYZsq{}}\PY{l+s+s1}{train}\PY{l+s+s1}{\PYZsq{}}\PY{p}{)}
\PY{n}{plt}\PY{o}{.}\PY{n}{plot}\PY{p}{(}\PY{n}{test}\PY{p}{,} \PY{n}{label} \PY{o}{=} \PY{l+s+s1}{\PYZsq{}}\PY{l+s+s1}{test}\PY{l+s+s1}{\PYZsq{}}\PY{p}{)}
\PY{n}{plt}\PY{o}{.}\PY{n}{xlabel}\PY{p}{(}\PY{l+s+s1}{\PYZsq{}}\PY{l+s+s1}{Size of testing set}\PY{l+s+s1}{\PYZsq{}}\PY{p}{)}
\PY{n}{plt}\PY{o}{.}\PY{n}{ylabel}\PY{p}{(}\PY{l+s+s1}{\PYZsq{}}\PY{l+s+s1}{Accuray of Decision tree }\PY{l+s+s1}{\PYZsq{}}\PY{p}{)}
\PY{n}{plt}\PY{o}{.}\PY{n}{legend}\PY{p}{(}\PY{n}{loc}\PY{o}{=}\PY{l+s+s1}{\PYZsq{}}\PY{l+s+s1}{upper center}\PY{l+s+s1}{\PYZsq{}}\PY{p}{)}
\end{Verbatim}
\end{tcolorbox}

            \begin{tcolorbox}[breakable, size=fbox, boxrule=.5pt, pad at break*=1mm, opacityfill=0]
\prompt{Out}{outcolor}{13}{\boxspacing}
\begin{Verbatim}[commandchars=\\\{\}]
<matplotlib.legend.Legend at 0x224b0489e80>
\end{Verbatim}
\end{tcolorbox}
        
    \begin{center}
    \adjustimage{max size={0.9\linewidth}{0.9\paperheight}}{1ML_files/1ML_61_1.png}
    \end{center}
    { \hspace*{\fill} \\}
    
    Since we have a multilevel tree, our training acc reaches 1.0, observing
the testing line, we see that the its maximum point is when the testing
set is 2/10, thus we can reflect that the optimal size of the testing
set is a bit less than 1/3

    CHANGING RANDOM STATE

    \begin{tcolorbox}[breakable, size=fbox, boxrule=1pt, pad at break*=1mm,colback=cellbackground, colframe=cellborder]
\prompt{In}{incolor}{14}{\boxspacing}
\begin{Verbatim}[commandchars=\\\{\}]
\PY{n}{train} \PY{o}{=} \PY{p}{[}\PY{p}{]}
\PY{n}{test} \PY{o}{=} \PY{p}{[}\PY{p}{]}
\PY{n}{random\PYZus{}state} \PY{o}{=} \PY{p}{[}\PY{p}{]}

\PY{k}{for} \PY{n}{x} \PY{o+ow}{in} \PY{n+nb}{range}\PY{p}{(}\PY{l+m+mi}{1}\PY{p}{,} \PY{l+m+mi}{100} \PY{p}{,} \PY{l+m+mi}{1}\PY{p}{)}\PY{p}{:}
    \PY{n}{X\PYZus{}train}\PY{p}{,} \PY{n}{X\PYZus{}test}\PY{p}{,} \PY{n}{Y\PYZus{}train}\PY{p}{,} \PY{n}{Y\PYZus{}test} \PY{o}{=} \PY{n}{train\PYZus{}test\PYZus{}split}\PY{p}{(}\PY{n}{X}\PY{p}{,} \PY{n}{Y}\PY{p}{,} 
\PY{n}{test\PYZus{}size}\PY{o}{=} \PY{l+m+mf}{0.20}\PY{p}{,} \PY{n}{random\PYZus{}state}\PY{o}{=}\PY{n}{x}\PY{p}{)}
    \PY{n}{clf} \PY{o}{=} \PY{n}{tree}\PY{o}{.}\PY{n}{DecisionTreeClassifier}\PY{p}{(}\PY{n}{criterion} \PY{o}{=} \PY{l+s+s1}{\PYZsq{}}\PY{l+s+s1}{entropy}\PY{l+s+s1}{\PYZsq{}}\PY{p}{,} \PY{n}{max\PYZus{}depth} \PY{o}{=} \PY{k+kc}{None}\PY{p}{)}
    \PY{n}{random\PYZus{}state}\PY{o}{.}\PY{n}{append}\PY{p}{(}\PY{n}{x}\PY{p}{)}
    \PY{n}{clf} \PY{o}{=} \PY{n}{clf}\PY{o}{.}\PY{n}{fit}\PY{p}{(}\PY{n}{X\PYZus{}train}\PY{p}{,} \PY{n}{Y\PYZus{}train}\PY{p}{)}
    \PY{n}{Ypredicted\PYZus{}train} \PY{o}{=} \PY{n}{clf}\PY{o}{.}\PY{n}{predict}\PY{p}{(}\PY{n}{X\PYZus{}train}\PY{p}{)}
    \PY{n}{Ypredicted\PYZus{}test} \PY{o}{=} \PY{n}{clf}\PY{o}{.}\PY{n}{predict}\PY{p}{(}\PY{n}{X\PYZus{}test}\PY{p}{)}
    \PY{n}{test}\PY{o}{.}\PY{n}{append}\PY{p}{(}\PY{n}{accuracy\PYZus{}score}\PY{p}{(}\PY{n}{Y\PYZus{}test}\PY{p}{,} \PY{n}{Ypredicted\PYZus{}test}\PY{p}{)}\PY{p}{)}
    \PY{n}{train}\PY{o}{.}\PY{n}{append}\PY{p}{(}\PY{n}{accuracy\PYZus{}score}\PY{p}{(}\PY{n}{Y\PYZus{}train}\PY{p}{,} \PY{n}{Ypredicted\PYZus{}train}\PY{p}{)}\PY{p}{)}

\PY{n}{plt}\PY{o}{.}\PY{n}{plot}\PY{p}{(}\PY{n}{train}\PY{p}{,} \PY{n}{label}\PY{o}{=} \PY{l+s+s1}{\PYZsq{}}\PY{l+s+s1}{train}\PY{l+s+s1}{\PYZsq{}}\PY{p}{)}
\PY{n}{plt}\PY{o}{.}\PY{n}{plot}\PY{p}{(}\PY{n}{test}\PY{p}{,} \PY{n}{label} \PY{o}{=} \PY{l+s+s1}{\PYZsq{}}\PY{l+s+s1}{test}\PY{l+s+s1}{\PYZsq{}}\PY{p}{)}
\PY{n}{plt}\PY{o}{.}\PY{n}{xlabel}\PY{p}{(}\PY{l+s+s1}{\PYZsq{}}\PY{l+s+s1}{Size of random state}\PY{l+s+s1}{\PYZsq{}}\PY{p}{)}
\PY{n}{plt}\PY{o}{.}\PY{n}{ylabel}\PY{p}{(}\PY{l+s+s1}{\PYZsq{}}\PY{l+s+s1}{Accuray of Decision tree }\PY{l+s+s1}{\PYZsq{}}\PY{p}{)}
\PY{n}{plt}\PY{o}{.}\PY{n}{legend}\PY{p}{(}\PY{n}{loc}\PY{o}{=}\PY{l+s+s1}{\PYZsq{}}\PY{l+s+s1}{upper center}\PY{l+s+s1}{\PYZsq{}}\PY{p}{)}
\end{Verbatim}
\end{tcolorbox}

            \begin{tcolorbox}[breakable, size=fbox, boxrule=.5pt, pad at break*=1mm, opacityfill=0]
\prompt{Out}{outcolor}{14}{\boxspacing}
\begin{Verbatim}[commandchars=\\\{\}]
<matplotlib.legend.Legend at 0x224b0513ee0>
\end{Verbatim}
\end{tcolorbox}
        
    \begin{center}
    \adjustimage{max size={0.9\linewidth}{0.9\paperheight}}{1ML_files/1ML_64_1.png}
    \end{center}
    { \hspace*{\fill} \\}
    
    \begin{tcolorbox}[breakable, size=fbox, boxrule=1pt, pad at break*=1mm,colback=cellbackground, colframe=cellborder]
\prompt{In}{incolor}{ }{\boxspacing}
\begin{Verbatim}[commandchars=\\\{\}]

\end{Verbatim}
\end{tcolorbox}


    % Add a bibliography block to the postdoc
    
    
    
\end{document}
